\documentclass[a4, 12pt]{article}

\usepackage[utf8]{inputenc}
\usepackage[slovene]{babel}
\usepackage{graphicx}
\usepackage{amsmath}
\usepackage{amsfonts}
\usepackage{latexsym}
\usepackage{amssymb}
\usepackage{listings}

\begin{document}

\title{ Computing optimal islands  \\
  \large Kratek opis projekta}

\author{Avtorja: Matevž Kopač, Tinkara Žitko}

\maketitle

\section{Opis problema}
V ravnini imamo podano množico točk P. Problem se ukvarja z iskanjem takega konveksnega večkotnika - otoka, katerega oglišča so podmnožica množice P in ima največjo možno ploščino. Večkotnik je konveksen, če v njem ali na njegovem robu v celoti leži vsaka daljica med poljubnima dvema ogliščema. Iskani večkotnik mora biti prazen, torej znotraj večkotnika ne sme ležati nobena točka iz množice P. 

Tak otok iščemo iterativno po množici točk P. Najprej poiščemo vse prazne trikotnike, ki jih nato povečujemo, do največjih možnih praznih konveksnih večkotnikov in izmed dobljenih izberemo največjega.

Postopek na vsakem dobljenem trikotniku začnemo z izbiro največjega oglišča, ki ga označimo z \emph{p}  in povezave \emph{e}, ki leži nasproti največjega oglišča. S \emph{T(p, e)} torej označimo ploščino otoka, ki vsebuje \emph{p} kot največjo točko in vsebuje povezavo \emph{e}. Z dodajanjem povezav želimo povečati ploščino, brez da bi kršili katero izmed začetnih predpostavk o konveksnosti in praznosti množice. Če krajišči povezave \emph{e} označimo z \emph{q} in \emph{q'} , najprej predpostavimo, da naš največji večkotnik vsebuje povezavi \emph{e} in \emph{pq}, ter iz oglišča \emph{q} iščemo povezave, ki bi povečale ploščino trenutnega večkotnika. Nato pa dobljeni večkotnik povečujemo še z dodajanjem povezav med ogljišči \emph{p} in \emph{q'}. 
Za ohranjanje konveksnosti lahko upoštevamo načelo, s katerim se znebimo neustreznih točk. Če navidezno podaljšamo daljico \emph{q'q}, lahko iz množice potencialnih oglišč izločimo vsa oglišča, ki ležijo pod njo.

\section{Reševanje problema}

Za reševanje problema bova uporabila dinamično programiranje, izhajala bova iz spodaj navedene formule za ploščino, ki bo postopoma gradila večkotnik, 
ki bo ustrezal našim predpostavkam. Na koncu bova iz množice vseh ustreznih večkotnikov izbrala tistega z največjo ploščino,
 $${T(p,e)} =  \text{max area} ( \triangle pqq') + T(p,e' )$$ 
 pri pogojih, da je T(p,e´) definirana, torej da v njeni notranjosti ni drugih točk in da je kot med $e$ in $e'$  manjši od 180°. 
Na vrhu bova dodala tudi horizontalno premico $h_p$, ki bo potekala skozi oglišče $p$. Vzemimo na primer dve oglišči $q_i$ in $q_j$, ki ležita pod $h_p$. Če bo trikotnik $ \triangle(p,q_iq_j)$ vseboval kakšno izmed preostalih oglišč, potem $q_iq_j$ ne bo rob končnega večkotnika. Sedaj smo verjetno množico točk že močno zmanjšali, saj kandidati za oglišča ležijo pod $h_p$ in nad $q'q$. Točke znotraj tega območje bova označila s $KO(E)$, kjer je E nov podaljšek roba trikotnika.

Problem lahko prevedemo na iskanje roba $q_iq_j$ z največjo uteženostjo s pomočjo dinamičnega programiranja. Pomagala si bova s problemom iz članka Computing optimal islands, ki ga bova prilagodila najinim zahtevam. Preštela bova točke, ki jih vsebuje večkotnik, ki ga ustvarimo z dodano povezano in izločila tiste, katerim se število spremeni več kot za 1 glede na prejšnji večkotnik, saj bi to pomenilo, da je še neka točka v notranjosti. 

\section{Poskusi z manipulacijo podatkov}

V nadaljevanju bova v programskem jeziku Python napisala program, ki s pomočjo dinamičnega programiranja poišče množico vseh ustreznih večkotnikov in izbere največjega. Izvedla bova poskus, kako povečanje točk vpliva na čas izračuna in ploščino največjega konveksnega večkotnika.
\end{document}

